% Adding additional packages
My favourite equation in physics is the tensor equation that forms the backbone of general relativity :
\subsection{Einstein Field Equations(EFE)} 

\begin{equation}
{  G_{\mu \nu}  } + {  \Lambda g_{\mu \nu}  } = {\kappa T_{\mu \nu}}
\label{eqn:main}
\end{equation}

\subsection{Description}

Einstein field equations [equation-\ref{eqn:main}], published by Albert Einstein as part of his General theory of relativity \cite{Einstein}, relates the geometry of space-time to the energy,momentum and stress quantity within it. 

Exact solutions to the equation can only be found with simplifying assumptions, like symmetry. Special cases of exact solutions are extensively studied since they model many gravitational phenomena like rotating black holes and the expanding universe. 

At points far away from gravitational sources, the space-time curvature can be taken to have only small deviations from flat space-time. This assumption can be used to convert equation-\ref{eqn:main} to a linear form. Linearized EFE is used to model gravitational waves.

\subsection{Terms}

\begin{enumerate}
    \item $ G_{\mu \nu} $ : Known as the Einstein tensor, defined as $ R_{\mu \nu} - {1 \over 2}R \, g_{\mu \nu} $ . $ R_{\mu \nu} $ is the Ricci curvature tensor and $ R $ is the scalar curvature. 
    \item $ \Lambda g_{\mu \nu} $ : $ \Lambda $ is the cosmological  constant and $ g_{\mu \nu} $ is the metric tensor. Together with the Einstein tensor, the LHS of equation-\ref{eqn:main} represents the curvature of space-time.
    \item $ \Lambda $ : Referred to as the cosmological constant, it was initially introduced by Einstein to accommodate a steady state universe. At present, $ \Lambda $ is taken as a positive value to account for the accelerating expansion of the universe.
    \item $ \kappa T_{\mu \nu} $ : $ \kappa $ is called the Einstein gravitational constant. $$ \kappa = \frac{8 \pi G }{ c^4} \label{eqn:2}$$ where $ G $ is the Newtonian constant of gravitation and $ c $ is the speed of light in vacuum. $  T_{\mu \nu} $ is the stress - energy tensor. The RHS of equation-\ref{eqn:main} thus represents the stress-energy-momentum content of the space-time.
\end{enumerate}

\begin{thebibliography}{1}

\bibitem[Einstein, A.]{Einstein} Einstein, A. (1916). Die      grundlage der allgemeinen relativit atstheorie, Annalen der Physik, 354(7):769–822

\end{thebibliography}
